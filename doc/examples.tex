\documentclass[a4paper,11pt]{article}

\usepackage{html}

\title{JOTM Examples guide}
\author{Jeff Mesnil}
\date{\today}

\begin{document}

\maketitle

\begin{abstract}
  This guide describes examples provided with JOTM. It explains the
  scenarii of the examples, how to setup and run them.
\end{abstract}

\noindent Updated May 25, 2005 by Roger Perey\\

\tableofcontents

\section{Available Examples}
\label{sec:into}
Five examples are available with JOTM:
\begin{itemize}
\item The first example, \hyperref{basic example}{\textbf{basic} (section
  }{)}{sec:basic_example}, has some interesting features such as how to
 configure JOTM to have clients accessing the same Transaction
 Manager using RMI/JRMP or RMI/IIOP.
\item The second example, \hyperref{jdbc example}{\textbf{jdbc} (section
  }{)}{sec:jdbc_example}, explains how to use JOTM and JDBC to provide
  distributed transactions with a database.
\item The third example \hyperref{JDBC-DIST example}{\textbf{jdbc-dist} (section
  }{)}{sec:jdbc-dist_example}, is JDBC-DIST or JDBC over distributed databases.
 This section explains how to configure and use JDBC over multiple databases.
\item The fourth example, \hyperref{JMS example}{\textbf{jms} (section
  }{)}{sec:jms_example}, explains how to use JOTM with a JMS provider (JORAM)
  supporting distributed transactional messages for your applications.
\item The fifth example, \hyperref{Tomcat example}{\textbf{tomcat} (section
  }{)}{sec:tomcat_example}, explains how to use JOTM with
  Tomcat to provide distributed transactions with a database from a
  Servlet or a JSP.

\end{itemize}

\section{Basic Example}
\label{sec:basic_example}

All Ant commands are executed from the \texttt{JOTM\_HOME/examples/basic/}
directory of a JOTM \emph{distribution} (examples do not work from the JOTM 
\emph{source} directory).

\subsection{Basic Scenario}
\label{sec:basic_scenario}


The \textbf{BASIC} example is a simple example using a Transaction
 Manager. The client application (\texttt{BasicExample} class)
looks up the \texttt{UserTransaction}, and builds two transactions:
\begin{itemize}
\item The first transaction is a simple begin/commit
\item the second transaction is also a simple begin/commit but is
  rolled back due to a timeout expiration (the BasicExample thread sleeps
 longer than the transaction timeout set for this transaction)
\end{itemize}


\subsection{Basic example Setup and compilation}
\label{sec:basic_setup}

To set up this example:
\begin{itemize}
\item create a name server (either a RMI registry or a CORBA nameserver)
\item invoke JOTM as the Transaction Manager providing a \texttt{UserTransaction class} utilizing JNDI
\item start \texttt{BasicExample} as the client application
\item The Ant tool is required to compile and run all example tests
\end{itemize}

\noindent To compile the Basic example, in \texttt{JOTM\_HOME/examples/basic/} directory, type
\begin{verbatim}
  ant compile
\end{verbatim}

\subsection{Running the Basic example}
\label{sec:basic_run}
First, set \texttt{JOTM\_HOME} to the directory of your JOTM distribution (e.g.,
\texttt{../jotm/output/dist}) from CVS.
\begin{verbatim}
UNIX
  export JOTM\_HOME=<JOTM_distribution_directory>
WINDOWS
  Set JOTM\_HOME=<JOTM_distribution_directory>
\end{verbatim}

To run the Basic example, change directory to the \texttt{JOTM\_HOME/examples/basic/}
directory.\\
The example may be run with two different configurations for
protocol communication:
\begin{itemize}
\item JOTM accessible through \textbf{RMI/JRMP} (default configuration)
\item JOTM accessible through \textbf{RMI/IIOP}
\end{itemize}

\noindent The two commands for executing the Basic example are:
\begin{verbatim}
  ant run.rmi.jrmp
  ant run.rmi.iiop
\end{verbatim}

\noindent The first assumes that JOTM is accessible through RMI/JRMP and
that an RMI registry is running on default port 1099.\\
The second assumes that JOTM is accessible through RMI/IIOP and
that a CORBA name server is running on port 19751.\\
Both assume the \texttt{UserTransaction} object is
accessible on JNDI with the name \texttt{UserTransaction}.\\

\noindent Note that these two targets are using the same
class, \texttt{BasicExample}. They differ in their settings: one is
for RMI/JRMP communication, the other is an RMI/IIOP client.

\subsubsection{Basic example using RMI/JRMP}
To run the Basic example using RMI/JRMP, enter (in a single line)
 within directory \texttt{JOTM\_HOME/lib/},
\begin{verbatim}
UNIX
  rmiregistry -J-classpath -Jjotm.jar:jotm_jrmp_stubs.jar 
  -J-Djava.security.policy=../conf/java.policy &
WINDOWS
  rmiregistry -J-classpath -Jjotm.jar;jotm_jrmp_stubs.jar
  J-Djava.security.policy=../conf/java.policy
\end{verbatim}
Then in \texttt{JOTM\_HOME/lib/}, (in a single line) type
\begin{verbatim}
UNIX
  java -classpath jotm.jar:jotm_jrmp_stubs.jar:../conf/ 
  org.objectweb.jotm.Main -u UserTransaction &
WINDOWS
  java -classpath jotm.jar;jotm_jrmp_stubs.jar; 
  ../conf/ org.objectweb.jotm.Main -u UserTransaction

\end{verbatim}


\noindent In the \texttt{JOTM\_HOME/examples/basic/} directory, type
\begin{verbatim}
  ant run.rmi.jrmp
\end{verbatim}
\noindent NOTE: By default JOTM is configured to run with RMI/JRMP so no modification
of the \texttt{JOTM\_HOME/conf/carol.properties} file is needed to run the 
BASIC example using RMI/JRMP).

\subsubsection{Basic example using RMI/IIOP}
To run basic example using RMI/IIOP,  change the settings of JOTM
to activate RMI/IIOP support : \
in \texttt{JOTM\_HOME/conf/carol.properties}
\begin{itemize}
\item set \texttt{carol.protocols} to \texttt{iiop}
\end{itemize}
(RMI/IIOP is activated and is the default protocol)\\

\noindent Then type
\begin{verbatim}
UNIX
  tnameserv -ORBInitialPort 19751 &
WINDOWS
  tnameserv -ORBInitialPort 19751
\end{verbatim}
Then in \texttt{JOTM\_HOME/lib/}, type (in a single line)
\begin{verbatim}
UNIX
  java -classpath jotm.jar:jotm_iiop_stubs.jar:../conf/ \
  org.objectweb.jotm.Main -u UserTransaction &
WINDOWS
  java -classpath jotm.jar;jotm_iiop_stubs.jar; \
  ../conf/ org.objectweb.jotm.Main -u UserTransaction 
\end{verbatim}

\noindent In the \texttt{JOTM\_HOME/examples/basic/} directory, type
\begin{verbatim}
  ant run.rmi.iiop
\end{verbatim}

\subsection{BASIC example Output}
\label{sec:basic_output}
Regardless of configuration, the output of Basic example is
the same :
\begin{verbatim}
$ ...
$
$ [java] create initial context
$ [java] lookup UserTransaction at : UserTransaction
$
$ [java] a simple transaction wich is committed:
$ [java]        - initial status : STATUS_NO_TRANSACTION
$ [java]        - after begin status : STATUS_ACTIVE
$ [java]        - after commit status : STATUS_NO_TRANSACTION
$
$ [java] a simple transaction which is rolled back.
$ [java] we set a transaction timeout to 1 second, begin the
$ [java] transaction, and wait 5 seconds before commiting it:
$ [java]        - initial status : STATUS_NO_TRANSACTION
$ [java]        - after begin status : STATUS_ACTIVE
$ [java]        - wait for 5 seconds
$ [java]        - after rollback status : STATUS_NO_TRANSACTION
$
$ [java] Basic example is OK.
$
$ ...
\end{verbatim}

\noindent If the message \texttt{"Basic example is OK."}is output, the
basic example is working. If not, check the JOTM
settings. Usually, trouble is due to incorrect settings
(clients try to access JOTM through RMI/JRMP expecting an RMI registry
 but find a CORBA name server, i.e. RMI/IIOP instead...).

\section{JDBC Example}
\label{sec:jdbc_example}

JOTM may be used with any database (utilizing a JDBC driver) providing
distributed transactional access to a databases.\newline
The JDBC example is a simple example using JTA
transactions with XAPool to provide transactional access to a database.
(Configuration files for PostgreSQL and MySQL are included with JDBC Example).\\

\noindent NOTE: Ant commands are executed from \texttt{JOTM\_HOME/examples/jdbc/}
directory of a JOTM \emph{distribution}. (Examples do not execute from the JOTM
\emph{source} directory). 

\subsection{Scenario}
\label{sec:jdbc_scenario}

JDBC Example Setup requires:
\begin{itemize}
\item a database is configured and started 
\item an RMI registry is started
\item the \texttt{JdbcExample} object is started
\end{itemize}
\noindent Jdbc Example starts a \texttt{DatabaseHelper} object with an embedded JOTM.
In turn, this sets up the JDBC objects. (i.e. \texttt{java.sql.Connection} with XAPool
from a configuration file) and sets JOTM as the transaction manager. This also binds
 \texttt{UserTransaction} in JNDI.
\begin{itemize}
\item \texttt{JdbcExample} prints a table from the database (without
  transaction)
\item a transaction is started by \texttt{UserTransaction}
\item an update statement is sent to the database
\item the transaction is completed (either committed or rolled back)
\item \texttt{JdbcExample} prints the table from the database (without
  transaction)  
\end{itemize}

\subsection{Setup, compilation, and execution}
\label{sec:jdbc_setup}

Before starting the example, the database must be properly
configured and the database server started.
The JDBC example works with any database that provides a JDBC
driver. JDBC example uses XAPool to take care of transactional
behaviors for JDBC objects.
JDBC example setup is explained for MySQL. For another database, it should be
straightforward to configure it properly.

\subsubsection{Database setup}
\label{sec:jdbc_db_setup}

The JDBC example set up expects:
\begin{itemize}
\item a database named \texttt{javatest}
\item a user with a login \texttt{mojo} and password \texttt{jojo}
(if a different user/password is used, the database properties file must change)
\item a \emph{transactional} table named \texttt{testdata} described as following:\\
  \begin{tabular}[t]{|c|c|}
    \hline ID & FOO \\
    \hline 1 & 1 \\
    \hline
  \end{tabular}
  with 
  \begin{itemize}
  \item \texttt{id} being an \texttt{int} (\emph{primary key})
  \item \texttt{foo} an \texttt{int}
  \end{itemize}
\end{itemize}

\noindent For example, on MySQL enter:
\begin{verbatim}
mysql> GRANT ALL PRIVILEGES ON *.* TO mojo
    ->   IDENTIFIED BY 'jojo' WITH GRANT OPTION;
mysql> create database javatest;
mysql> use javatest;
mysql> create table testdata (
    ->   id int not null auto_increment primary key,
    ->   foo int)type=InnoDB;
mysql> insert into testdata values(null, 1);
\end{verbatim}
NOTE: Do not forget to set \texttt{testdata} type to \texttt{InnoDB} to enable
transaction support.\\

Database configuration is stored in a database properties file (e.g. \texttt{mysql.properties} 
or \texttt{postgresql.properties}) containing the following:

\begin{itemize}
\item \texttt{driver} - Name of the JDBC driver
\item \texttt{url} - URL to connect to the data base
\item \texttt{login} - user login
\item \texttt{password} - user password 
\end{itemize}

\subsubsection{JDBC Example Compilation}
In \texttt{JOTM\_HOME/examples/jdbc/} directory, type
\begin{verbatim}
     ant compile
\end{verbatim}
to compile the example

\subsection{Run the example}
\label{sec:jdbc_run}
Set \texttt{JOTM\_HOME} to the directory of your JOTM distribution (e.g.,
\texttt{.../jotm/output/dist}) from CVS.

\noindent To execute the JDBC example, first check that RMI protocol is
activated (\texttt{../../conf/carol.properties} \texttt{carol.protocols}
 is set to \texttt{jrmp}.) \\
To start a RMI registry on default port 1099, in \texttt{JOTM\_HOME/lib/} directory enter 
 (in a single line):
\begin{verbatim}
UNIX
     rmiregistry -J-classpath -Jjotm.jar:jotm_jrmp_stubs.jar \
     -J-Djava.security.policy=../conf/java.policy &
WINDOWS
     rmiregistry -J-classpath -Jjotm.jar;jotm_jrmp_stubs.jar \
     J-Djava.security.policy=../conf/java.policy
\end{verbatim}

\noindent Set the classpath:
\begin{verbatim}
UNIX 
     export CLASSPATH=../../lib/jotm.jar:../../lib/jotm_jrmp_stubs.jar\
     :../../lib/xapool.jar:../../conf:.:$JDBC_JARS \
WINDOWS 
     Set CLASSPATH=../../lib/jotm.jar; \
     ../../lib/jotm_jrmp_stubs.jar\; \
     ../../lib/xapool.jar;../../conf/;.;%JDBC_JARS% 
\end{verbatim}
where \texttt{JDBC\_JARS} is the location of the JDBC driver jar
file(s). For example:
\begin{itemize}
\item \texttt{pg73jdbc2.jar} for PostgreSQL 7.3
\item \texttt{mysql-connector-java-2.0.14-bin.jar} for MySQL 3.23
\end{itemize}
The latest JAR files may be downloaded from
\begin{itemize}
\item \url{http://www.mysql.com/downloads/api-jdbc-stable.html/}
\item \url{http://jdbc.postgresql.org/download/}
\end{itemize}

\subsection{Executing the JDBC example}
\label{sec:jdbc_exec}
\noindent For example;
\begin{verbatim}
     java JdbcExample postgresql commit 2
\end{verbatim}
\noindent sets \texttt{foo} value to 2 and commits the transaction using
PostgreSQL
\noindent Another example;
\begin{verbatim}
     java JdbcExample mysql rollback 0
\end{verbatim}
sets \texttt{foo} value to 0 and rollback the transaction using MySQL

\subsubsection{Using JDBC Example}
\begin{verbatim}
     java JdbcExample [database] [completion] [number]
\end{verbatim}
where
\begin{itemize}
\item \texttt{database} is
  \begin{itemize}
  \item \texttt{postgresql} or
  \item \texttt{mysql} 
\end{itemize}
\noindent To determine a installed database, the JDBC example looks
 for a configuration file with a name \texttt{[database].properties}
  
\item \texttt{completion} is
  \begin{itemize}
  \item \texttt{commit} to commit the transaction or,
  \item \texttt{rollback} to rollback the transaction
  \end{itemize}
\item \texttt{number} must be an integer
\end{itemize}
\subsection{JDBC example output}
\label{sec:jdbc_output}

For command line:
\begin{verbatim}
     java JdbcExample postgresql commit 2
\end{verbatim}
the ouptut is 
\begin{verbatim}
:
start server

postgresql configuration:
-- listing properties --
login=mojo
url=jdbc:postgresql://localhost/javatest
password=jojo
driver=org.postgresql.Driver
------------------------

create initial context
lookup UserTransaction at : UserTransaction
get a connection
before transaction, table is:
        id      foo
        1       0
begin a transaction
update the table
*commit* the transaction
after transaction, table is:
        id      foo
        1       2
close connection
stop server
JDBC example is ok.
\end{verbatim}
As shown above, the transaction is \emph{committed} and \texttt{foo} value
is set to 2 in the database.

\noindent Another command example:
\begin{verbatim}
     java JdbcExample mysql rollback 3
\end{verbatim}
outputs:
\begin{verbatim}
:
start server

mysql configuration:
-- listing properties --
login=mojo
url=jdbc:mysql://localhost/javatest
password=jojo
driver=org.gjt.mm.mysql.Driver
------------------------

create initial context
lookup UserTransaction at : UserTransaction
get a connection
before transaction, table is:
        id      foo
        1       1
begin a transaction
update the table
*rollback* the transaction
after transaction, table is:
        id      foo
        1       0
close connection
stop server
JDBC example is ok.
\end{verbatim}
In this example the value of \texttt{foo} is not changed in the database
because the transaction is \emph{rolled back}.

\section{JDBC-DIST Example}
\label{sec:jdbc-dist_example}

The JDBC Distributive example is a simple example showing how to utilize
 JTA transactions with two databases

\subsection{JDBC-DIST example scenario}

Much of the scenario for the JDBC-DIST example follows the same procedure
 as the simple JDBC example detailed in Section 3.

\begin{itemize}
\item 	A database server is configured and started. In this case, 
two databases are utilized, Javatest1 and Javatest2.
\item An RMI registry is started (RMI/JRMP)
\item The JdbcDistExample object is started
\end{itemize}
\noindent JdbcExample starts a DistDatabaseHelper object with an embedded JOTM.
 DistDatabaseHelper sets up the JDBC objects (i.e. java.sql.Connection with
 XAPool per the configuration file), and sets JOTM as the transaction manager.
 UserTransaction also binds to JNDI.
\begin{itemize}
\item JdbcDistExample asks DistDatabaseHelper for two connections, updates
 a table in two different databases within a transaction and completes the
 transaction based on parameters entered by the user.
\item Update statements are sent to the databases
\item Transactions are completed (either committed or rolled back)
\item JdbcDistExample prints the database tables (without transaction)
\end{itemize}

\subsection{JDBC-DIST example setup, compilation, and execution}

The JDBC-DIST example expects:
\begin{itemize}
\item A database named javatest for each host name. If localhost is defined for use,
 two databases named javatest1 and javatest2 are required.
\item The SQL commands in the java code reference databases javatest1 and javatest2 
\item A user with login "mojo" and password "jojo"
\item A *transactional* table in each database named testdata as outlined in the
 previous JDBC example
\end{itemize}
\noindent Database configurations are stored in properties files.
 In this example mysql1.properties and mysql2.properties are used. Another
 example may use postgresql1.properties and postgresql2.properties, and/or
 oracle1.properties and oracle2.properties. 

\noindent Each database.properties file contains the following:
\begin{itemize}
\item driver - Name of the JDBC driver
\item url - URL to connect to the data base
\item login - user login
\item password - user password
\end{itemize}

\subsection{JDBC-DIST Compilation}

In JOTM\_HOME/examples/jdbc-dist/ directory, to compile the JDBC-DIST example type:
\begin{verbatim}
     ant compile
\end{verbatim}

\subsection{JDBC-DIST Execution}
Before executing JDBC-DIST example, check that RMI protocol is
activated (in JOTM\_HOME/conf/carol.properties file, carol.rmi.activated
is set to jrmp). 

\noindent In JOTM\_HOME/lib/ directory, enter (on a single line): 
\begin{verbatim}
UNIX:
  rmiregistry -J-classpath -Jjotm.jar:jotm_jrmp_stubs.jar: 
  commons-cli-1.0.jar:connector-1_5.jar:howl.jar: 
  mysql-connector-java-3.1.6-bin.jar -J-Djava.security.policy=../conf/java.policy & 
WINDOWS: 
  rmiregistry -J-classpath -Jjotm.jar;jotm_jrmp_stubs.jar; 
  commons-cli-1.0.jar;connector-1_5.jar;howl.jar; 
  mysql-connector-java-3.1.6-bin.jar -J-Djava.security.policy=../conf/java.policy 
\end{verbatim}

The above starts a RMI registry on default port 1099.

\subsection{JDBC-DIST Classpath}
In JOTM\_HOME/lib, on one line enter:

\begin{verbatim}
UNIX:
    export CLASSPATH=../../lib/jotm.jar: 
    ../../lib/jotm_jrmp_stubs.jar:../../lib/xapool.jar:../../conf: 
    ../../lib/connector-1_5.jar:../../lib/howl.jar: 
    ../../lib/commons-cli-1.0.jar:../../lib/mysql-connector-java-3.1.7-bin.jar:
    .:../../conf/ & 
WINDOWS: 
    set CLASSPATH=%CLASSPATH%;
    ../../lib/jotm.jar;../../lib/jotm_jrmp_stubs.jar;
    ../../lib/xapool.jar;../../lib/mysql-connector-java-3.1.7-bin.jar;
    ../../lib/jdbc.jar;../../lib/connector-1_5.jar;../../lib/howl.jar;
    ../../lib/commons-cli-1.0.jar;./;../../conf/
\end{verbatim}

\noindent NOTE: If user or password is other then defined above, or if names other
 than JAVATEST1/2 are used:
\begin{itemize}
\item change database1.properties and database2.properties password/user
 to reflect database user/password 
\item change database1.properties and database2.properties 
to reflect name1 and name2
\end{itemize}

\subsection{Starting JDBC-DIST example}
Some examples of JDBC-DIST commands:
\begin{verbatim}
UNIX:
// set table value to 4 and commit the transaction using Oracle 
java JdbcDistExample oracle1 oracle2 commit 4 
// set table value to 2 and commit the transaction using PostgreSQL 
java JdbcDistExample postgresql1 postgres2 commit 2 
// set table value to 0 but rollback the transaction using MySQL 
java JdbcDistExample mysql1 mysql2 rollback 0
// set table value to 2 and commit the transaction using MySQL 
java JdbcDistExample mysql1 mysql2 commit 2 &

WINDOWS:
// set table value to 4 and commit the transaction on Oracle
java JdbcDistExample oracle1 oracle2 commit 4
// set table value to 2 and commit the transaction on PostgreSQL
java JdbcDistExample postgresql1 postgres2 commit 2 
// set table value to 0 but rollback the transaction on MySQL
java JdbcDistExample mysql1 mysql2 rollback 0 
// set table value to 2 and commit the transaction using MySQL 
java JdbcDistExample mysql1 mysql2 commit 0
\end{verbatim}

\subsection{Examples Mixing JDBC-DIST databases}
Set table value to 0 on Oracle and rollback the transaction on MySQL
\begin{verbatim}
UNIX:
     java JdbcDistExample oracle1 mysql1 rollback 0 & 
WINDOWS:
     java JdbcDistExample oracle1 mysql1 rollback 0
\end{verbatim}

\subsection{JDBC-DIST execution}
\begin{verbatim}
UNIX:
     java JdbcExample [database1] [database2] [completion] [number] &
WINDOWS:
     java JdbcExample [database1] [database2] [completion] [number]
\end{verbatim}

Where:
Database[] equals:
\begin{itemize}
\item oracle
\item postgresql 
\item mysql 
\end{itemize}
NOTE: JdbcExample looks for a configuration file name [database].properties
Completion equals one of the following:
\begin{itemize}
\item commit
\item rollback

\item Number = an integer value
\end{itemize}

\subsection{JDBC-DIST example output}
Output is similar to the JDBC example with reports for each database\\


\section{JMS Example}
\label{sec:jms_example}

Any JMS (\htmladdnormallink{Java Message Service}{http://java.sun.com/products/})
 provider may be used with JOTM to gain advantage of both 
\emph{message-oriented architecture} and \emph{distributed transactions}.\\
The JMS example uses JORAM (\url{http://joram.objectweb.org/}) as the
JMS provider.\\

\noindent All Ant commands are executed from the \texttt{JOTM\_HOME/examples/jms/}
directory of a JOTM \emph{distribution}. (Examples do not work from the JOTM
\emph{source} directory).

\subsection{JMS Scenario}
\label{sec:jms_scenario}
The \textbf{JMS} example explains how to use JOTM with a JMS provider (JORAM)
 to provide distributed transactional messages.

\begin{itemize}
\item The RMI registry is started
\item JOTM is started with \texttt{UserTransaction} and
  \texttt{TransactionManager} objects accessible through JNDI
\item An application (\texttt{SimpleJmsXa} class) starts JORAM, sets up
  the JMS objects (\texttt{Queue}, \texttt{Session},
  \texttt{ConnectionFactory}) and registers them in JOTM as XA
  resources
\item  The application starts a message sender, 
  \texttt{SimpleSender}, and a message receiver, \texttt{SimpleReceiver}
\end{itemize}

\noindent \texttt{SimpleSender} sends 4 messages to a JMS queue:
\begin{itemize}
\item one outside a transaction
\item one inside a transaction with a commit result
\item one inside a transaction with a rollback result
\item and a final message with special text to stop SimpleReceiver
\end{itemize}

\noindent \texttt{SimpleReceiver} receives 3 messages from the same JMS queue:
\begin{itemize}
\item the first message outside a transaction
\item second message inside a transaction with a commit result
\item the last last message (with the special text)
\end{itemize}

\noindent NOTE: SimpleReceiver does not receive the 3rd (sent) message
  because it is rolled back.

\subsection{Setup and compilation}
\label{sec:jms_setup}

To compile the JMS example, in the JOTM\_HOME/examples/jms directory type:
\begin{verbatim}
    ant compile
\end{verbatim}

\subsection{Executing the JMS example}
\label{sec:jms_run}
Set \texttt{JOTM\_HOME} to the directory of your JOTM distribution (e.g.,
\texttt{../jotm/output/dist}) from CVS.
\begin{verbatim}
UNIX
  export JOTM\_HOME=<JOTM_distribution_directory>
WINDOWS
set JOTM\_HOME=<JOTM_distribution_directory>
\end{verbatim}

\noindent To execute the JMS example, in \texttt{JOTM\_HOME/lib/} type: 
\begin{verbatim}
UNIX
  rmiregistry -J-classpath -Jjotm.jar:jotm_jrmp_stubs.jar  
  -J-Djava.security.policy=../conf/java.policy &
WINDOWS
  rmiregistry -J-classpath -Jjotm.jar;jotm_jrmp_stubs.jar 
  J-Djava.security.policy=../conf/java.policy
\end{verbatim}

In \texttt{JOTM\_HOME/lib/}, type (on one line):
\begin{verbatim}
UNIX
  java -classpath jotm.jar:jotm_jrmp_stubs.jar:../conf/  
  org.objectweb.jotm.Main -u UserTransaction -m TransactionManager&
WINDOWS
  java -classpath jotm.jar;jotm_jrmp_stubs.jar;../conf/ 
  org.objectweb.jotm.Main -u UserTransaction -m TransactionManager
\end{verbatim}

\noindent In the \texttt{JOTM\_HOME/examples/jms/} directory, type
\begin{verbatim}
  ant run.jms
\end{verbatim}

\noindent Since the client application of the JMS example is a simple
RMI/JRMP client, use the default protocol configuration for JOTM (i.e
RMI/JRMP) in \texttt{JOTM\_HOME/conf/carol.properties} file.
\subsection{JMS Example Output}
\label{sec:jms_output}

the output of the \textbf{jms} example:
\begin{verbatim}
$ ...
$
$ [java] [SimpleJmsXa] lookup the TransactionManager.
$ [java] [SimpleJmsXa] start the JMS server.
$ [java] [SimpleJmsXa] JMS server started.
$ [java] [SimpleJmsXa] create JMS objects, register them in JOTM and bind them.
$ [java] [SimpleJmsXa] JMS objects available.
$ [java] [SimpleJmsXa] start simple sender.
$ [java] [SimpleSender] send : non transactional message
$ [java] [SimpleSender] send : transactional message with commit
$ [java] [SimpleSender] send : transactional message with rollback
$ [java] [SimpleSender] send : LAST message
$ [java] [SimpleJmsXa] start simple receiver.
$ [java] [SimpleReceiver] received: non transactional message
$ [java] [SimpleReceiver] received: transactional message with commit
$ [java] [SimpleReceiver] received: LAST message
$ [java] [SimpleJmsXa] JMS server stopped
$
$ ...
\end{verbatim}

\noindent If \texttt{SimpleSender} has sent 4 messages and
\texttt{SimpleReceiver} has received 3 messages, then
the \textbf{jms} example is working correctly.


\section{Tomcat Example}
\label{sec:tomcat_example}

JOTM may be integrated with Tomcat providing distributed transactional access
 to resources from Servlets or JSP. \\
\noindent Using the JDBC example, Tomcat works with any database providing a JDBC driver.
The Tomcat example uses XAPool to handle transactional behaviors and the 
pooling of JDBC objects.

\subsection{Tomcat example Scenario}
\label{sec:tomcat_scenario}

The scenario of the Tomcat example is simple and is based on the example
provided by Tomcat in JNDI Datasource HOW-TO with the addition of some 
transaction code.\\

\noindent The user sends a request to a JSP file (\texttt{test.jsp}) requesting 
\emph{commit} or \emph{rollback} and the incrementation of an integer stored in
 a database.
The JSP delegates the JDBC and transaction code to a JavaBean
(\texttt{foo.DBTest} class).\\

\noindent The code of \texttt{foo.DBTest} is simple:
\begin{itemize}
\item a JDBC \texttt{Connection} is created from a \texttt{DataSource} retrieved through JNDI
\item a \texttt{UserTransaction} is retrieved from JNDI
\item a transaction is started
\item the value of the integer \texttt{foo} stored in the
  database is read (SQL query)
\item increment the value of \texttt{foo} by 1 in the database (SQL
  update)
\item depending on the choice of the user (\texttt{commit} or
  \texttt{rollback}), the transaction is either \emph{committed} or
  \emph{rolled back}
\item \texttt{foo} value from the database (SQL
  query) is re-read and displayed using the JSP.
\end{itemize}

\subsection{Tomcat example Setup and Compilation}
\label{sec:tomcat_setup}


\subsubsection{Tomcat setup}

The Tomcat example uses Tomcat downloaded from
\url{http://jakarta.apache.org}. 

\noindent There's no setup for Tomcat. Unzip Tomcat to a directory to use it.

\subsubsection{Tomcat example Database setup}

The database setup for the Tomcat example is the same as the JDBC example. 
Please refer to the \hyperref{Database setup}{Database setup (section
  }{)}{sec:jdbc_db_setup} of the JDBC example.

\noindent Copy the JDBC driver jar file for your database into the \texttt{common/lib/}
 directory of Tomcat.
\subsubsection{Compilation}
\label{sec:tomcat_compilation}

In \texttt{JOTM\_HOME/examples/tomcat/} directory, type

\begin{verbatim}
  ant war
\end{verbatim}
This compiles Java files and creates a WAR file
(\texttt{examples/tomcat/output/dbtest.war}) containing the web 
application and all that is required to use JOTM. 

\subsubsection{Tomcat example Deployment}
Copy the \texttt{output/dbtest.war} file just created, to the
 \texttt{webapps/} directory of Tomcat.\\

\noindent Copy \texttt{JOTM\_HOME/examples/tomcat/dbtest.xml} to the
\texttt{webapps/} and \texttt{conf/catalina/localhost directories of Tomcat}.\\
The dbtest.xml file describes the context associated with your web
 application. In this file, the properties are set to access your database:
\begin{itemize}
\item \texttt{driverClassName} - Name of the JDBC driver
\item \texttt{url} - URL to connect to the data base
\item \texttt{username} - user login
\item \texttt{password} - user password 
\end{itemize}

NOTE: By default, \texttt{dbtest.xml} is configured to use PostgreSQL
 as its database. 
To use another database (i.e. MYSQL), change the properties of this file 
(specifically the driverClassName and url).\\

\noindent \texttt{dbtest.xml} also describes the resource factories 
(JDBC and Transaction) used by your web application.

\subsubsection{JOTM jar files}
Copy the following JOTM jar files, located in the \texttt{JOTM\_HOME/lib/}
 directory of JOTM, to the common/lib/ directory of Tomcat.

\begin{itemize}
\item \texttt{jotm.jar}
\item \texttt{jotm\_jrmp\_stubs.jar}
\item \texttt{carol-2.0.5.jar}
\item \texttt{jta-spec1\_0\_1.jar}
\item \texttt{jts1\_0.jar}
\item \texttt{commons-logging.jar}
\item \texttt{log4j.jar}
\item \texttt{objectweb-datasource.jar}
\item \texttt{xapool-1.5.0.jar}
\item \texttt{howl-0.1.8.jar}
\end{itemize}

\noindent Create a file named carol.properties in the common/classes/ 
directory of Tomcat with the following properties.
\begin{verbatim}
# JNDI (Protocol Invocation)
carol.protocols=jrmp

# Local RMI Invocation
carol.jvm.rmi.local.call=true

# do not use CAROL JNDI wrapper
carol.start.jndi=false

# do not start a name server
carol.start.ns=false

# Naming Factory
carol.jndi.java.nameing.factory.url.pkgs=org.apache.nameing
\end{verbatim}

\subsection{Executing the Tomcat example}
\label{sec:tomcat_run}

Change directory to the \texttt{bin/} directory of Tomcat and enter:
\begin{verbatim}
UNIX
  ./catalina.sh run
WINDOWS
   startup
\end{verbatim}

Use your favorite browser to go to the URL
\begin{verbatim}
http://localhost:8080/dbtest/test.jsp
\end{verbatim}

\noindent Choose if you want to \texttt{commit} or \texttt{rollback} the 
value of the integer and click on the completion button.  
If you've chosen \texttt{commit}, the integer value displayed on the page should
increment by one. If \texttt{rollback} is chosen, the integer value displayed
on the page should stay the same (no increment). 

\subsection{Integration of JOTM and Tomcat}
\label{sec:tomcat_jotm_integration}

For a more technical explanation on the integration of JOTM in Tomcat,
please refer to Tomcat/JOTM HOW-TO. 

\section{Contacts}
\label{sec:contacts}

If you have trouble making the examples work or want to
contribute to JOTM, please contact us (\url{mailto:jotm@objectweb.org}).
\end{document}
