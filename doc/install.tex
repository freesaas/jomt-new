\documentclass[a4paper,11pt]{article}

\usepackage{html}

\title{JOTM Installation Guide}
\author{Jeff Mesnil}
\date{\today}

\begin{document}

\maketitle

\begin{abstract}
  This guide describes how to install JOTM.
\end{abstract}

\noindent Updated May 5, 2005 by Roger Perey

\tableofcontents

\section{Introduction}
\label{sec:intro}
This guide describes how to install and configure JOTM.\\
JOTM (Java Open Transaction Manager) is a fully functional
Open Source implementation of JTA
(\htmladdnormallink{Java Transaction API} {http://java.sun.com/products/jta/}).\\
JOTM is a mature
project in use for several years with the JOnAS
application server project as the JOnAS transaction
manager. \\

\noindent If you have any questions or comments on JOTM, do not
hesitate to let us know. (\url{mailto:jotm@objectweb.org})
\section{Prerequisites}
\label{sec:prereq}
In order to install or use JOTM, a few common tools are required.

\begin{itemize}
\item A Java platform
\item The Ant tool for compiling and running the examples
  provided with JOTM
\item Additionally, for executing the JOTM test suite
  , the Junit Framework is required.
\end{itemize}
\subsection{Java}
\label{sec:java_prereq}
JOTM requires a Java compiler (for executing the \texttt{javac} command),
 and a Java runtime environment (for executing the \texttt{java} command).
\begin{itemize}
\item Sun Java Development Kits are available at \url{http://java.sun.com/j2se/}
\item IBM Java Development Kits are available at \url{http://www.ibm.com/java/jdk/}
\end{itemize}

\emph{Presently, JOTM is built and tested using JDK 1.3 and 1.4}
\subsection{Ant}
\label{sec:ant_prereq}
JOTM uses Ant for build processes and for executing the JOTM
examples.\\
Download Ant from \url{http://jakarta.apache.org/ant/}.
Please read the Ant documentation for Ant set up (\url{http://jakarta.apache.org/ant/manual/}).

\subsection{Junit}
\label{sec:junit_prereq}
Junit is required by JOTM as the framework for executing the JOTM
test suite (\url{http://junit.org/}).\\
If interested in executing JOTM test suite, download Junit from the following web site 
and copy the supplied \texttt{junit.jar} file to the \texttt{lib/} directory
 of the Ant installation.

\section{Installing JOTM}
\label{sec:install}
JOTM may be downloaded and installed with three possible configurations:
\begin{itemize}
\item as a \emph{source package} including source, jars, and examples (test cases)
\end{itemize}

\noindent \texttt{jotm-x.y-src.tgz} (UNIX/LINUX), \texttt{jotm-x.y-src.zip} (WINDOWS)
where \texttt{x.y} is the
 version of JOTM (referenced as JOTM \emph{source package})
\begin{itemize}
\item as a \emph{distribution package} jars and examples only (no source)
\end{itemize}
\texttt{jotm-x.y.tgz} \texttt{jotm-x.y.zip} (WINDOWS) 
referenced as JOTM \emph{distribution package}
\begin{itemize}
\item from CVS (similar to the source package)
\end{itemize}

\subsection{From a source or distribution package}
\label{sec:pack_install}
Download JOTM packages from the JOTM download page
(\url{http://www.objectweb.org/jotm/download/}).\\
To install JOTM from a package, unzip the file with \textbf{gunzip}
and \textbf{tar} on Unix systems and \textbf{winzip} on Windows
 creating a new directory \texttt{jotm-x.y}.

\subsection{From CVS}
\label{sec:cvs_install}
CVS provides network transparent source control for developers. 
(For more information see \url{http://www.gnu.org/software/cvs}).\\
CVS allows other commands (read only) i.e. \textbf{cvs status} or
\textbf{cvs diff}.\\
To download JOTM using CVS, enter:\\
NOTE: Windows users should use a DOS window.
\begin{verbatim}
cvs -d:pserver:anonymous@cvs.forge.objectweb.org:/cvsroot/jotm login 
$ CVS password: [type Enter]
cvs -z3 
-d:pserver:anonymous@cvs.forge.objectweb.org:/cvsroot/jotm co -P jotm 
\end{verbatim}

This creates a working repository of JOTM in the \texttt{jotm/}
directory

\subsection{Test suite}
\label{sec:test_install}
To test JOTM, download Junit 
(see \hyperref{JUnit}{section}{}{sec:junit_prereq}) and the 
JOTM Test Suite Guide. Dounloading the JOTM test suite is only possible using CVS.\\
To download the test suite, within the \textbf{same} directory used for checking out
 JOTM, enter:\\

\begin{verbatim}
  $ cvs -d:pserver:anonymous@cvs.forge.objectweb.org:/cvsroot/jotm co -P test
\end{verbatim}
This creates a working repository of JOTM Test Suite in the
\texttt{test/} directory (at the same level as the \texttt{jotm/} directory).\\
For more information, see ``JOTM Test Suite Guide'' (\url{http://www.objectweb.org/jotm/doc/}).

\subsection{Web site}
\label{sec:web_install}
JOTM web site (\url{http://www.objectweb.org/jotm/}) is constructed
from XML pages kept in the JOTM CVS repository.\\
To download JOTM Web, from the same directory used for checking out JOTM, enter:\\

\begin{verbatim}
$ cvs --d:pserver:anonymous@cvs.forge.objectweb.org:/cvsroot/jotm co -P web
\end{verbatim}
This creates a working repository of JOTM web site in the
\texttt{web/} directory.\\
For more info, see ``ObjectWeb Website Guideline''
(\url{http://www.objectweb.org/architecture/infrastructure/guideline/}).


\section{Ant commands}
\label{sec:ant_cmds}
After downloading JOTM as a source package (or from CVS), 
build JOTM structures (create a JOTM distribution before use.)\\
JOTM relies on Ant for the build process.\\
All Ant commands are enterd in the \texttt{jotm/} directory
(i.e. the same directory containing the \texttt{build.xml} file).\\
To obtain a list and description of Ant available targets for JOTM,
type:
\begin{verbatim}
  ant -projecthelp
\end{verbatim}

\subsection{Compile and build JOTM}
\label{sec:build_cmd}
This step is necessary if you have installed only source files 
(e.g. downloaded using CVS or from a source package) or if JOTM source code is 
 modified.\\
Enter:
\begin{verbatim}
  ant dist
\end{verbatim}
This causes a compile of all JOTM source files and creates all jar files.\\
A working version of JOTM is now available in the
\texttt{output/dist/} subdirectory (the \emph{distribution} directory
of JOTM).

\subsection{Generate Javadoc}
\label{sec:jdoc_cmd}
To generate JOTM Javadoc, simply type:
\begin{verbatim}
  ant jdoc
\end{verbatim}
This creates browsable Javadoc pages in the \texttt{output/dist/jdoc/}
 subdirectory.


\subsection{Generate Documentation}
\label{sec:doc_cmd}
JOTM documentation is written in LaTeX.\\
The \textbf{pdflatex} tool generates PDF files and
\textbf{latex2html} generates HTML files. These tools may
not be installed on your system.\\
PDF generation is triggered by the \texttt{pdflatex}
command. For HTML generation a shell
script, \textbf{doc2html}, is used. NOTE: At this time, HTML documentation
generation is only available on Linux. Documentation is 
available online at \url{http://www.objectweb.org/jotm/doc/} in
both PDF and HTML formats).
\begin{itemize}
\item If \textbf{pdflatex} is installed on your system, 
generate PDF documentation by entering
\begin{verbatim}
  ant -Dpdflatex=1
\end{verbatim}
\item For HTML documentation \textbf{latex2html} and working on \emph{Linux}, 
generate HTML documentation by typing
\begin{verbatim}
  ant -Dlatex2html=1
\end{verbatim}
\end{itemize}

\noindent For both PDF and HTML documentation, enter
\begin{verbatim}
  ant -Dpdflatex=1 -Dlatex2html=1
\end{verbatim}

\noindent Generated documents are placed in the \texttt{output/dist/doc} directory.

\subsection{Clean JOTM}
\label{sec:clean_cmd}

To remove files generated during the compilation or build process, enter:
\begin{verbatim}
  ant clean
\end{verbatim}
This deletes the directory path named \textbf{output} (created by ant -dist).

\section{Project Structure}
\label{sec:proj_desc}

JOTM may be downloaded under two different \emph{conditions}:
\begin{itemize}
\item as a \emph{source} (from a \emph{source} package or CVS)
\item as a \emph{distribution} (from a \emph{distribution} package)
\end{itemize}

\subsection{Source Structure}
\label{sec:source_desc}

When downloading JOTM source, the structure of the \texttt{jotm/}
directory is:
\begin{itemize}
\item \texttt{GettingStarted.txt} - a text file specifying the main Ant targets
  for JOTM
\item \texttt{README.txt} - README file
\item \texttt{build.xml} - the main Ant build file used for building JOTM
\item \texttt{build.properties} - configuration of Ant properties 
\item \texttt{archive/} - manifest files for JOTM jar files 
\item \texttt{bin/} - empty directory 
\item \texttt{doc/} - all documentation source files
  (e.g. \texttt{install.tex} used to generate this guide)
\item \texttt{examples/} - the examples source code
\item \texttt{ext/} - the extensions used by JOTM. 
  Composed of properties files for configuration
\item \texttt{externals/} - libraries for building and using JOTM
\item \texttt{src/} - JOTM source code

\end{itemize}

\subsection{Distribution Structure}
\label{sec:dist_desc}

After building JOTM (i.e. the \texttt{output/dist/}
subdirectory) or if downloaded from a \emph{distribution} package,
the distribution structure of JOTM follows:
\begin{itemize}
\item \texttt{GettingStarted.txt} - a text file containing the main Ant targets
  for JOTM
\item \texttt{README.txt} - README file
\item \texttt{build.xml} - the main Ant build file used for building JOTM
\item \texttt{build.properties} - configuration of Ant properties
\item \texttt{bin/} - empty directory
\item \texttt{archive/} - manifest files for JOTM jars 
\item \texttt{doc/} - this directory may contain PDF and HTML version of the
  various documentation files as well as the documentation source 
(see \hyperref{Generate Documentation}{section}{}{sec:doc_cmd})
\item \texttt{examples/} - the same directory as in the \emph{source}
  structure. Compile the examples before running them.
\item \texttt{src/} - JOTM source code
\item \texttt{output/} - This directory is created when you run the ANT 
build and dist commands.
\end{itemize}
Subdirectories are created under /output for running JOTM examples:
\begin{itemize}
\item \texttt{output/build} - Contains the class files and 
stubs needed by JOTM
\item \texttt{output/dist} - has five subdirectories
\item \texttt{output/dist/conf} - holds the properties 
files for JOTM execution
\item \texttt{output/dist/doc} - holds created documentation 
(PDF and HTML files)
\item \texttt{output/dist/jdoc/} - Javadoc of JOTM
\item \texttt{output/dist/lib/} - JOTM library directory (it contains both 
JOTM jar files and other jar files)
\item \texttt{output/dist/examples} - holds the subdirectories 
explained in the JOTM Examples Guide.

\end{itemize}

\section{Running JOTM}
\label{sec:jotm_run}
Once the \emph{distribution} of JOTM is built, it's available for use.\\
For a simple test, change to the \texttt{lib/} directory of your JOTM
distribution and enter:
\begin{verbatim}
  java -classpath jotm.jar org.objectweb.jotm.Main --help
\end{verbatim}
A help message should return from JOTM\\
Congratulations! JOTM is working!\\


\subsection{Log configuration}
\label{sec:log_conf}
Since JOTM version 1.4,  JOTM has used
\htmladdnormallink{Commons Logging}{http://jakarta.apache.org/commons/} and 
\htmladdnormallink{Log4J}{http://jakarta.apache.org/log4j/} 
as its logging system.\\
Log configuration is stored in a file, \texttt{log4j.properties},
located in the \texttt{conf/} directory of your JOTM distribution.\\

\noindent For more information on Commons Logging and Log4J, see
\htmladdnormallink{Commons Logging Javadoc}{http://jakarta.apache.org/commons/logging/api/index.html}
and \htmladdnormallink{Log4J documentation}{http://jakarta.apache.org/log4j/docs/documentation.html}.

\section{Examples}
\label{sec:examples}
Once you have a \emph{distribution} of JOTM, a test to determine if
installation is correct, is to run the JOTM examples (in the
\texttt{examples/} directory of the distribution).\\
Examples execute from the \texttt{examples/} directory of a
JOTM \emph{distribution} (Examples do not work from the \texttt{examples/}
directory of JOTM \emph{source}).\\

\noindent For more information about Examples ine 
JOTM distribution, read the ``JOTM Examples Guide''
(\url{http://www.objectweb.org/jotm/doc/}).

\section{Contacts}
\label{sec:contacts}

If you have trouble installing JOTM, questions, or if you want
to contribute, please contact us at
(\url{mailto:jotm@objectweb.org}).
\end{document}
